\documentclass[11pt]{amsart}
\usepackage{geometry}                % See geometry.pdf to learn the layout options. There are lots.
\geometry{letterpaper}                   % ... or a4paper or a5paper or ... 
%\geometry{landscape}                % Activate for for rotated page geometry
%\usepackage[parfill]{parskip}    % Activate to begin paragraphs with an empty line rather than an indent
\usepackage{graphicx}
\usepackage{amssymb}
\usepackage{epstopdf}
\DeclareGraphicsRule{.tif}{png}{.png}{`convert #1 `dirname #1`/`basename #1 .tif`.png}

\usepackage{xcolor}
\definecolor{magicmint}{rgb}{0.67, 0.94, 0.82}
\title{NuCypher: Economic Protocol}
\author{The Author}
%\date{}                                           % Activate to display a given date or no date

\begin{document}
\pagecolor{magicmint!30}
\maketitle

\section{Context: The Service Provided by the NuCypher Network}

From a user's point of view, the NuCypher network provides a {\it decentralized access control service}. Users may include consumer-facing applications (e.g. a medical record management app), platforms (e.g. a marketplace for mobile browsing data), internet infrastructure (e.g. a database-as-a-service) and other multi-user systems (e.g. an intra-vehicle data sharing device). 
\\
\\
NuCypher network nodes (hereafter:{\it workers}) provide this service to users by facilitating sharing flows. This involves updating the permission(s) associated with data payloads. In some ways, the service is similar to that of a large-scale key management system (e.g. AWS KMS), but critically, the sharing flow is end-to-end encrypted and thus does not imply trust in the security or integrity of a centralized third-party or custodian. A single permission update typically involves multiple workers, each of whom only handles fragments of keys and ciphertexts, such that a unilateral refusal to execute one's assignment is almost always inconsequential to the end-user. 
\\
\\
Nonetheless, workers are expected to behave properly, and are compensated for their efforts as access managers. In practice, this job involves being reliably online at all times, securely holding onto sharing policies and their corresponding {\it re-encryption key fragments}, and performing {\it re-encryptions} (i.e. permission updates) in response to legitimate access requests. This work is remunerated via fees (denominated in ETH), paid by the aforementioned users. 
\\
\\
Initially, payouts from fees alone are unlikely to incentivize a sufficient number of would-be workers to join the network. Though the costs of performing re-encryptions are very low, in the early stages of the network, so too will the profits derived purely from fees. Hence, until demand for decentralized access control has reached a stable growth phase, an inflation-based subsidy will be necessary to motivate workers. 
\\
\\
Although workers need not be trusted with private data, given that the objects they hold do not give them this power or permission, they can still misbehave: refusing to re-encrypt, producing incorrect re-encryptions, being offline (hence risking the disruption of a sharing flow), and, though extrordinarily difficult to orchestrate, colluding with other nodes and the designated recipient(s) to obtain a data owner’s private key. To disincentivize misbehaviour, in particular the first two offenses, and reward good behaviour, workers are required to stake a native token ({\it NU}). This allows them to participate in the network and receive access control jobs. Stakes function as a collateral, with one's initial deposit worth a non-trivial sum. Moreover, access control jobs are assigned proportionally to the size of a worker's stake, relative to the aggregate of all current stakes. Similarly, a worker's inflation payout depends on the relative size of their stake – with coefficients based on other factors such as the length of their commitment to the network. 

%\subsection{}


\appendix

% The slashing protocol
% !TEX root = mint-paper.tex

\section{The slashing protocol}
\subsection{Verifying proofs of re-encryption correctness}

Let the input $capsule$ be the tuple $(E,V,s)$. For each $cFrag = (E_1, V_1, id, X_A)$ with associated proof $\pi = (E_2, V_2, U_2, U_1, kfragSignature, \rho, metadata)$.
The validation of a re-encryption correctness proof consists on the following steps:
 \begin{enumerate}
    \item Check that $kfragSignature$ is correct. 
    \item Compute the hash value $h = H(E, E_1, E_2, V, V_1, V_2, U, U_1, U_2, metadata)$
    \item Check that the following equations hold:
    \begin{align} 
\rho \cdot E &\stackrel{?}{=} E_2 + h \cdot E_1  			\label{eq:proof-e} \\ 
\rho \cdot V &\stackrel{?}{=} V_2 + h \cdot V_1 			\label{eq:proof-v} \\
\rho \cdot U &\stackrel{?}{=} U_2 + h \cdot U_1				\label{eq:proof-u}
\end{align}
 \end{enumerate}

\subsection{On-chain approaches to re-encryption correctness validation}

\subsubsection{Approach A: No ECC arithmetic support}

As the time of writing, there is no native support for curve \textsf{secp256k1} arithmetic in the EVM. 
This implies that if we want to perform on-chain EC arithmetics (e.g., point addition, point doubling, scalar multiplication, etc.), we need to write a new library. This was our first approach, which culminated with the development of Numerology, our own Solidity library for \textsf{secp256k1} arithmetics. 

Numerology provides optimized ECC algorithms useful for verifying Schnorr-like zero-knowledge proofs (which  usually involve checking equations of the form $aP + bQ = R$). These algorithms are honed to spend as little gas as possible. In order to do so, we implement a method for simultaneous multiplication, instead of computing $aP$ and $bQ$ independently. We further speed up each multiplication by using an efficient curve endomorphism of \textsf{secp256k1}; to this regard, we implement the Gallant-Lambert-Vanstone (GLV) method, which allows trading each 256-bit multiplication, to a simultaneous multiplication with two 128-bit scalars. As a result, using Numerology to evaluate an equation of the form $aP + bQ = R$ takes approximately 500,000 gas. 

Our re-encryption verification equations can be checked using Numerology once rearranged. For example, Equation \ref{eq:proof-e} can be rewritten as $\rho \cdot E - h \cdot E_1 = E_2$. Hence, the total cost in gas for the three equations is approximately 1.5 million gas, although with some additional optimizations due to the reuse of $\rho$ and $h$ in the tree of them, the total cost can be reduced to 1.3 million gas. 

\subsubsection{Approach B: Using ``EC Multiplication verification`` as a primitive}
Clearly, spending 1.3 million gas for a single REC verification is far from ideal. 

Let us assume that we there is an inexpensive function \textsf{multVer}$(a, P, Q)$ that allows to verify whether $a \cdot P = Q$. 
Let $P_{\rho} = \rho \cdot E$ and $P_{h} = h \cdot E_1$ be points precomputed off-chain by the verifier. Then Equation \ref{eq:proof-e} can be rewritten as $P_{\rho} = E_2 + P_{h}$, and it can be verified according to the following procedure.
\begin{enumerate}
	\item Check that \textsf{multVer}$(\rho, E, P_{\rho})$ and \textsf{multVer}$(h, E_1, P_h)$ hold; otherwise, output $\mathsf{Error}$.
	\item Output $\mathsf{True}$ if $P_{\rho} = E_2 + P_{h} $ holds; $\mathsf{False}$, otherwise.
\end{enumerate}






\end{document}  